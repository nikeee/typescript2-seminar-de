Diese Ausarbeitung beschäftigt sich mit den neuen Features in allen Bereichen von \ts 2, vergleichend mit \ts 1. Vorallem in größeren Software-Projekten gibt es bei \js-Anwendungen Probleme bei der Wartbarkeit und Fehleranfälligkeit, weshalb sich in dort der Einsatz von \ts als vorteilhaft erweist. Die Neuerungen von \ts soll die Fehlererkennung und die Abbildung des \js-Typensystems zur Entwicklungs- und Kompilierzeit im Vergleich zum Vorgänger verbessern.  Der Fokus liegt auf den Features der Sprache ab Version 2.0 bis einschließlich 2.2, der zum Zeitpunkt des Erscheinens dieser Ausarbeitung aktuellsten Version. Die Programmiersprache wurde um neue Schlüsselwörter und Typen erweitert, sowie syntaktisch auf den Stand der neuen \esss gebracht. Es wird auch das Ökosystem, das sich um die Sprache herum gebildet hat, betrachtet. Der Compiler wird mit Optionen ausgeliefert, mit denen die Emittierung von \js sowie die Typüberprüfung genauer gesteuert werden kann.

Da sich \ts fortlaufend in Entwicklung befindet, gibt diese Ausarbeitung auch einen Ausblick auf die Zukünftigen Pläne.
Nach Betrachtung der Neuerungen kommt diese Ausarbeitung zum dem Schluss, dass die Migration bestehender \js- oder \ts-Projekte die Codestabilität verbessert.